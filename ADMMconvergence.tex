\documentclass{article}
\linespread{1.2}
\usepackage[margin = 1.25 in]{geometry}
\usepackage{wrapfig}
\usepackage{amsfonts}
\usepackage[utf8]{inputenc}
\usepackage[T1]{fontenc}
\usepackage{graphicx}
\usepackage[english]{babel}
\usepackage[algoruled]{algorithm2e}

\renewcommand{\theequation}{\thesection.arabic{equation}}

\renewcommand{\thefigure}{\thesection.\arabic{figure}}



\renewcommand{\vec}[1]{\mathbf{#1}}
\renewcommand{\theequation}{\thesubsection.\arabic{equation}}
\DeclareGraphicsExtensions{.pdf,.png,.jpg, .gif}

\usepackage{amsthm}

\usepackage[english]{babel}
\usepackage{mathtools}

%\usepackage[OT2,T1]{fontenc}
%\DeclareSymbolFont{cyrletters}{OT2}{wncyr}{m}{n}
%\DeclareMathSymbol{\sha}{\mathalpha}{cyrletters}{"58}

\DeclareFontFamily{U}{wncy}{}
\DeclareFontShape{U}{wncy}{m}{n}{<->wncyr10}{}
\DeclareSymbolFont{mcy}{U}{wncy}{m}{n}
\DeclareMathSymbol{\Sh}{\mathord}{mcy}{"58} 
\DeclareMathOperator*{\argmin}{arg\,min}

\newcounter{eqn}
\renewcommand*{\theeqn}{\alph{eqn})}
\newcommand{\num}{\refstepcounter{eqn}\text{\theeqn}\;}

\makeatother
\newcommand{\vectornorm}[1]{\left|\left|#1\right|\right|}
\newcommand*\conjugate[1]{\bar{#1}}

\newtheorem{thm}{Theorem}
\newtheorem{defn}{Definition}
 %\theoremstyle{plain}
  \newtheorem{theorem}{Theorem}[section]
  \newtheorem{corollary}[theorem]{Corollary}
  \newtheorem{proposition}[theorem]{Proposition}
  \newtheorem{lemma}[theorem]{Lemma}
\newtheorem{example}[theorem]{Example}
  \newtheorem{definition}[theorem]{Definition}
  \newtheorem{conj}[theorem]{Conjecture}
 \newtheorem{condition}{Condition}
 \newtheorem{remark}[theorem]{Remark}

\newcommand{\supp}{\operatorname{supp}} 
\newcommand{\vc}[1]{{\mathbf{ #1}}}
\newcommand{\tn}{\widetilde{\nabla}_{n} }
\newcommand{\Z}{{\mathbb{Z}}}
\newcommand{\re}{{\mathbb{R}}}
\newcommand{\II}{{\mathbb{I}}}
\newcommand{\ep}{{\mathbb{E}}}
\newcommand{\pr}{{\mathbb{P}}}
\newcommand{\FF}{{\mathcal{F}}}
\newcommand{\TT}{{\mathcal{T}}}
\newcommand{\phin}{\phig{n}}
\newcommand{\phig}[1]{\phi^{(#1)}}
\newcommand{\ol}[1]{\overline{#1}}
\newcommand{\eff}{{\rm eff}}
\newcommand{\suc}{{\rm suc}}
\newcommand{\tends}{\rightarrow \infty}
\newcommand{\setS}{{\mathcal{S}}}
\newcommand{\setP}{{\mathcal{P}}}
\newcommand{\setX}{{\mathcal{X}}}
\newcommand{\nec}{{\rm nec}}
\newcommand{\bd}{{\rm bd}}

\title{The convergence of (Centralised)( ADMM}

\begin{document}

\section{Preliminaries and Notation}

One such algorithm is the alternating direction method of multipliers \cite{Boyd2010a}, (ADMM). This algorithm solves problems of the form

\begin{align}
\argmin_{x} f\left( x \right) + g\left(z\right) \nonumber
\\
\text{s.t } Ux +Vz = c
\label{admm}
\end{align}

where \(f\) and \(g\) are assumed to be closed convex functions with range in \(\re\), \(U \in \re^{p \times n}\) and \(V\in \re^{p \times m}\) are matrices (not assumed to have full rank), and \(c \in \re^p\).

ADMM consists of iteratively minimising the augmented Lagrangian 

\begin{align*}
L_p\left(x, z, \eta\right) = f\left( x\right) +& g\left(z\right)+\eta^T\left(Ux+Vz-c\right) + \frac{\rho}{2}\|Ux+Vz-c\|_2^2
\label{admm_form}
\end{align*}

(\(\eta\) is a Lagrange multiplier), and \(\rho\) is a parameter we can choose to make \(g(z)\) smooth \cite{nesterov2005smooth}, with the following iterations:

\begin{align}
x^{k+1} &:= \argmin_{x} L_\rho\left(x,z^k,\eta^k\right)\\
z^{k+1} &:= \argmin_{z} L_\rho\left(x^{k+1},z,\eta^k\right)\\
\eta^{k+1} &:= \eta^{k} + \rho \left(Ux^{k+1} + Vz^{k+1} - c\right)
\label{admm_algo}
\end{align}

The alternating minimisation works because of the decomposability of the objective function: the \(x\) minimisation step is independent of the \(z\) minimisation step and vice versa.  

We will work frequently with the \textit{dual} of problem \eqref{admm} which is

\begin{equation}
D\left(\lambda\right) = \argmax_{\lambda} -F^*\left(A^Tx\right) + \langle \lambda, c \rangle - G^*\left(B^t\lambda\right)
\end{equation}

where \(f^*\) is the convex conjugate:

\begin{equation}
F^*\left(p\right) = \sup_p \langle x,p \rangle - f\left(x\right)
\end{equation}

For example the convex conjugate of the \(l_1\) norm \( g\left(x\right) = \abs{x} \) is:

\begin{equation}
f^*\left(x^*\right) =  \twopartdef { 0 } {\abs{x} \leq 1} {\infty} {\abs{x} > 1}
\end{equation}

One way to measure how far the iterations of ADMM are from the optimal solution is to define the primal and dual residuals:

\begin{align}
r_k &= c - Ax_k - Bz_k \\
d_k &= \rho A^T B \left(x_k - x_{k-1}\right)
\end{align}

\section{Convergence of ADMM}

\begin{theorem}
Consider the ADMM iterations defined by \eqref{admm}. Supposing that both \(f,g\) are strongly convex and that:

\begin{equation}
\rho^3 \leq \frac{\sigma_f \sigma_g }{\tau\left(A^TA\right)\tau\left(B^TB\right)}
\end{equation}

Then the sequence of iterates \(\{\eta_k\}\) satisfy:

\begin{equation}
D\left(\eta^*\right) - D\left(\eta\right) \leq \frac{\vectornorm{\eta^* - \eta_1}}{2\rho(k-1)}
\end{equation}
\label{convergence}
\end{theorem}

Before we prove theorem \ref{convergence} we need a couple of lemmas.

\begin{definition}
$$\Psi\left(\lambda\right) = A \nabla F^*\left(A^T\lambda\right)$$
$$\Phi\left(\lambda\right) = B\nabla G^*\left(B^T\lambda\right)$$
\end{definition}

Note that the derivative of \( D\left(\lambda \right) \) is \(b - \Phi - \Psi \), and maximising the dual problem is equivalent to finding a solution to \( c \in \Phi\left(\lambda^*\right) + \Psi\left(\lambda^*\right) \). Since \(f,g\) are both strongly convex, then \(\Psi, \Phi\) are both Lipchitz continuous with \(L_{\Psi} \leq \frac{\tau(A^TA)}{\sigma_f}\) and \(L_{\Phi} \leq \frac{\tau(B^TB)}{\sigma_g}\) 

\begin{lemma}
Let \(\lambda, z \in \re^n\), and define

\begin{align}
\lambda^{1/2} &= \lambda + \rho\left(c - Ax^{k+1} - Bz^{k}\right) \\
\lambda^{+} &= \lambda + \left(c - Ax^{k+1} - Bz^{k+1}\right)
\end{align}

Then we have 

\begin{align}
Ax^{k+1} &= A \nabla F^*\left(A^T \lambda^{1/2}\right) := \Psi\left(\lambda^{1/2}\right) \\
Bz^{k+1} &= B \nabla G^*\left(B^T \lambda^+\right) := \Phi\left(\lambda^+\right)
\end{align}
\end{lemma}

\begin{lemma}
Suppose that

\begin{equation}
\rho^3 \leq \frac{\sigma_f \sigma_g }{\tau\left(A^TA\right)\tau\left(B^TB\right)}
\end{equation}
and that \(Bz = \Phi(\lambda)\). Then for any \(\gamma \in \re^n\):

\begin{equation}
D\left(\lambda^+\right) - D\left(\gamma\right) \geq \rho^{-1} \langle \gamma- \lambda, \lambda - \lambda^+ \rangle + \frac{1}{2\rho}\vectornorm{\lambda - \lambda^+}^2
\end{equation}
\end{lemma}

Proof
\begin{proof}[Theorem \eqref{convergence}]

\end{proof}

\end{document}