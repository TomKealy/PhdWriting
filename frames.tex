\documentclass{article}
\bibliographystyle{plain}
\input{macros}

\title{Frames}

\begin{document}
\maketitle

Given a set of observations described by a linear model:

\begin{equation}
y = Ax + n 
\end{equation}

where \(y \in \re^{m}, x\in \re^n, A \in \re^{m \times n}, and n ~ N\left(0,1\right) \in \re^m\), we wish to recover \(x\) from these measurements. 

One approach is to form a basis, or an over-complete basis, from \(y\) as:

\begin{equation}
Q = yy^t
\end{equation}

if we let \(v = Ax\), we see that \(Q\) is approximately \(vv^t\). One way to form a solution is to find an \(\hat{x}\) which is maximally correlated with the \(k\) largest eigenvectors of \(Q\).

I.e we can express \(\hat{x}\) as:

\begin{equation}
\hat{x} = \sum_{i=1}^k \lambda_i yy^t
\end{equation}

in particular \(v\) is a \( \vectornorm{v}^2 \) eigen-vector of \(vv^t\).

One way to find such an \(\hat{x}\) would be to form the matrix \(\textbf{U}\) by taking the \(k\) largest eigen-vectors of \(Q\) normalised by their respective eigen-values, and then correlating these eigen-vectors against the matrix \(A\). 

\subsubsection*{OMP}

\begin{algorithm}[H]

 \SetKwData{Left}{left}\SetKwData{This}{this}\SetKwData{Up}{up}
 \SetKwFunction{Match}{match}\SetKwFunction{Update}{update}
 \SetKwInOut{Input}{input}\SetKwInOut{Output}{output}

 \Input{Matrix \(\textbf{U}\), sensing matrix \(\textbf{A}\), stopping criterion \(s\)}
 initialization \\
 \(k \leftarrow 0\) \\
 \(\Lambda_0 = \emptyset\) \\
 \(\textbf{r}_k \leftarrow \textbf{U}\) \;
 \While{\(k \neq s\) or \(\textbf{r}_k \neq 0 \)}{
   \(k \leftarrow k+1\)\;
   \Match{\(h_k = A^T \textbf{r}^{k-1}\)}\;
   \(\Lambda_k = \Lambda_{k-1} \cup \argmin_j{h_k\left(j\right)} \)\;
  \Update:\;
  \(\hat{\textbf{x}} \leftarrow \textbf{A}_{\Lambda_k}^\dagger \textbf{U}\)\;
  \(\textbf{r}_k \leftarrow \textbf{U} - \textbf{A}_{\Lambda_k}^\dagger \textbf{U} \)
 }
 \Output{\(\hat{\textbf{x}} = \argmin_{x:\supp\left(x\right) \subseteq  \Lambda_k}\|U-Ax\|_2\)}
 \caption{Orthogonal Matching Pursuit}
\end{algorithm}

OMP iteratively detects the support set of the signal - proceeding by extending the estimate by a single index per iteration. The algorithm uses a matched filter to find the column which is maximally correlated with the current estimate, and solves a least-squares problem to update the residual vector.

\end{document}