\documentclass[11pt]{article}
\usepackage{amsmath, amssymb}
\usepackage{amsfonts}
\usepackage{amsthm}
\usepackage[algoruled]{algorithm2e}
\usepackage{titlesec}
\usepackage{graphicx}

\DeclareMathOperator\erf{erf}

\newcommand{\BigO}[1]{\ensuremath{\operatorname{O}\bigl(#1\bigr)}}
\renewcommand{\vec}[1]{\mathbf{#1}}

\newtheorem{example}{Example}[section]
\newtheorem{thm}{Theorem}[section]
\newtheorem{lem}{Lemma}[section]
\newtheorem{definition}{Definition}[section]
\newtheorem{cor}{Corollary}[section]

\begin{document}
\title{Soft Connection}
\author{Tom Kealy}

\date{December 18, 2013}

\maketitle

This is a short note to quickly justify some bounds on the power constant \(\beta\), needed to connect a rgg with soft connection function parametrised by \(\beta\). 

A result of Gupta and Kumar (Critical Power for connectivity in Wireless Networks) is that if all nodes radiate power to fill a disk of radius \(r\), then the network is asymptotically connected iff: 

\begin{equation}
\pi r^2 = \frac{log{n}+c\left(n\right)}{n}
\end{equation}

This implies that there is a radius \(r_{conn}\) above which the network is connected: 

\begin{equation}
r_{conn} = \sqrt{\frac{log{n}}{\pi n}}
\end{equation}
and the diameter of the connected component is

\begin{equation}
d = \Theta\left(\sqrt{\frac{\pi n}{log{n}}}\right)
\end{equation}

In the following we compare the rgg with soft connections to the equivalent rgg with hard connections: the minimal \(\beta\) we should choose, so that the (soft) graph is connected is the beta that puts a \(1 - \varepsilon\) ammount of mass in the (hard) circle around the nodes, as if it were a hard connected graph.

In the case of a Rayleigh distributed variable:

\begin{equation}
F\left(r\right) = \int_0^{r_\beta} 1 - e^{-\beta r^2} dr = r_\beta	- \frac{\sqrt{\pi} \erf{\left(r_\beta \sqrt{\beta}\right)}}{2\sqrt{\beta}} = 1 - \epsilon
\end{equation}

choose \(r_\beta \geq r_{conn}\) and the graph should be connected w.h.p. I.e \(r_\beta	 = r_{conn} + \delta\) (choose a circle so that most of the mass of the distribution is in the circle, but the radius ensures connection).

\begin{equation}
\sqrt{\frac{log{n}}{\pi n}} - \frac{\sqrt{\pi} \erf{r_\beta}\sqrt{\beta}}{2\sqrt{\beta}} = 1 - \epsilon
\end{equation}

This should allow you see how varying \(\beta\) changes the diameter of the connected component. 

\end{document}