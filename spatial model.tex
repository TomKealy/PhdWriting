\documentclass{article}
\usepackage{titlesec}
\makeatletter
\@addtoreset{section}{part}
\makeatother
\titleformat{\part}[display]
{\normalfont\LARGE\bfseries\centering}{}{0pt}{}

\linespread{1.2}
\usepackage[margin = 1.25 in]{geometry}
\usepackage{wrapfig}
\usepackage{amsfonts}
\usepackage[utf8]{inputenc}
\usepackage[T1]{fontenc}
\usepackage{graphicx}
\usepackage[english]{babel}
\usepackage[algoruled]{algorithm2e}

\renewcommand{\theequation}{\thesection.arabic{equation}}

\renewcommand{\thefigure}{\thesection.\arabic{figure}}



\renewcommand{\vec}[1]{\mathbf{#1}}
\renewcommand{\theequation}{\thesubsection.\arabic{equation}}
\DeclareGraphicsExtensions{.pdf,.png,.jpg, .gif}

\usepackage{amsthm}

\usepackage[english]{babel}
\usepackage{mathtools}

%\usepackage[OT2,T1]{fontenc}
%\DeclareSymbolFont{cyrletters}{OT2}{wncyr}{m}{n}
%\DeclareMathSymbol{\sha}{\mathalpha}{cyrletters}{"58}

\DeclareFontFamily{U}{wncy}{}
\DeclareFontShape{U}{wncy}{m}{n}{<->wncyr10}{}
\DeclareSymbolFont{mcy}{U}{wncy}{m}{n}
\DeclareMathSymbol{\Sh}{\mathord}{mcy}{"58} 
\DeclareMathOperator*{\argmin}{arg\,min}

\newcounter{eqn}
\renewcommand*{\theeqn}{\alph{eqn})}
\newcommand{\num}{\refstepcounter{eqn}\text{\theeqn}\;}

\makeatother
\newcommand{\vectornorm}[1]{\left|\left|#1\right|\right|}
\newcommand*\conjugate[1]{\bar{#1}}

\newtheorem{thm}{Theorem}
\newtheorem{defn}{Definition}
 %\theoremstyle{plain}
  \newtheorem{theorem}{Theorem}[section]
  \newtheorem{corollary}[theorem]{Corollary}
  \newtheorem{proposition}[theorem]{Proposition}
  \newtheorem{lemma}[theorem]{Lemma}
\newtheorem{example}[theorem]{Example}
  \newtheorem{definition}[theorem]{Definition}
  \newtheorem{conj}[theorem]{Conjecture}
 \newtheorem{condition}{Condition}
 \newtheorem{remark}[theorem]{Remark}

\newcommand{\supp}{\operatorname{supp}} 
\newcommand{\vc}[1]{{\mathbf{ #1}}}
\newcommand{\tn}{\widetilde{\nabla}_{n} }
\newcommand{\Z}{{\mathbb{Z}}}
\newcommand{\re}{{\mathbb{R}}}
\newcommand{\II}{{\mathbb{I}}}
\newcommand{\ep}{{\mathbb{E}}}
\newcommand{\pr}{{\mathbb{P}}}
\newcommand{\FF}{{\mathcal{F}}}
\newcommand{\TT}{{\mathcal{T}}}
\newcommand{\phin}{\phig{n}}
\newcommand{\phig}[1]{\phi^{(#1)}}
\newcommand{\ol}[1]{\overline{#1}}
\newcommand{\eff}{{\rm eff}}
\newcommand{\suc}{{\rm suc}}
\newcommand{\tends}{\rightarrow \infty}
\newcommand{\setS}{{\mathcal{S}}}
\newcommand{\setP}{{\mathcal{P}}}
\newcommand{\setX}{{\mathcal{X}}}
\newcommand{\nec}{{\rm nec}}
\newcommand{\bd}{{\rm bd}}
\begin{document}
\nocite{*}
\title{Spatial Model}
\date{\today}
\author{Tom Kealy}
\maketitle

\section{Model}
We are trying to sense and reconstruct a wideband signal, divided into \(L\) channels. We have a (connected) network of \(M\) (= 50) nodes placed uniformly at random within the square \(  \left[0,1\right]\times \left[0,1\right] \).

We write the power spectral density (psd) of the \(sth\) transmitter as:

\begin{equation}
\phi_s = \beta_{bs} \psi_b\left(f\right)
\label{basis_expansion}
\end{equation}
\\
with the convention that repeated indices are summed over. 

This model expresses in psd of the transmitter in a suitable basis - for example \(\psi_b\left(f\right)\) could be zero everywhere except for the set of frequencies where \(f=b\) i.e. \(\psi\) is a rectangular function with height \(\beta_{bs}\) and support \(f\). Other candidates for \(\psi\) include splines (e.g. raised cosines), and complex exponentials. 

Given this, the psd at the \(rth\) receiver is:

\begin{equation}
\phi_r = g_{sr}\phi_s = g_{sr}\beta_{bs}\psi_b\left(f\right)
\end{equation}

where

\begin{equation}
g_{sr} = \exp\left(-||x_r - x_s||_2^\alpha\right)
\end{equation}

is the channel response between the \(sth\) transmitter and the \(rth\) reciver.

We can write the psd at the rth receiver, over all \(k\) frequencies as:

\begin{equation}
\phi_{rk} = g_{sr}\phi_s 
\end{equation}

This model can be summarised using Kronecker products as follows:

\begin{equation}
\phi_{rk} = \left( g_{sr} \bigotimes \psi_{bk} \right)\left(\psi_{bk} \bigotimes \beta_{bs}\right) = g_{sr}\psi_{kb} \bigotimes \psi_{kb}\beta_bs
\end{equation}

\(\beta_{bs} \in \re^{1 \times n_s}\), \(g_{sr} \in \re^{n_r \times n_s}\) and \(\psi_{kb} \in 1 \times n_kn_b\) where \(n_k\) is the number of frequency bands (in this example \(n_k = n_b\).

For example

In the absence of knowledge of the location of the transmitters we introduce a grid of \textit{candidate} locations, to make the above model linear. \(s\) now runs over the set of these candidate locations.

The problem of estimating the coefficients, \(\beta\), from noisy observations \(y = \phi_r + N\left(0,1\right)\) is now one that can be tackled by linear regression/convex optimisation.

\section{Thoughts}
In the noiseless case we would like to calculate:

\begin{equation}
\pr\left(\text{obs at r} | \text{transmission from s}\right) = \pr(Obs | Tx)
\end{equation}

Applying Bayes formula:

\begin{equation}
\pr(Obs | Tx) = \frac{\pr(Tx | Obs) \pr(Obs)}{\pr(Tx)}
\end{equation}

From \ref{basis_expansion}, \(\pr(Tx | Obs)\) is \(g_{rs}\), and \(\pr(Tx)\) is \(|Tx|/8\) as reciever \(r\) can possible mistake transmitter \(s\) for any of it's 8 neighbours. \(\pr(Obs)\) is a Rayleigh random variable normalised by the number of receivers \(|Rx|\). 

\begin{equation}
\pr(Obs | Tx) = \exp\left(-||x_r - x_s||_2^\alpha\right) \exp(-\gamma ||x_r - x_s||_2^2) \frac{|Tx|}{8|Rx|}
\end{equation}

\end{document}