\documentclass{article}
\usepackage{titlesec}
\makeatletter
\@addtoreset{section}{part}
\makeatother
\titleformat{\part}[display]
{\normalfont\LARGE\bfseries\centering}{}{0pt}{}

\newcommand{\mf}{\psi\left(h\right)}

\linespread{1.2}
\usepackage[margin = 1.25 in]{geometry}
\usepackage{wrapfig}
\usepackage{amsfonts}
\usepackage[utf8]{inputenc}
\usepackage[T1]{fontenc}
\usepackage{graphicx}
\usepackage[english]{babel}
\usepackage[algoruled]{algorithm2e}

\renewcommand{\theequation}{\thesection.arabic{equation}}

\renewcommand{\thefigure}{\thesection.\arabic{figure}}



\renewcommand{\vec}[1]{\mathbf{#1}}
\renewcommand{\theequation}{\thesubsection.\arabic{equation}}
\DeclareGraphicsExtensions{.pdf,.png,.jpg, .gif}

\usepackage{amsthm}

\usepackage[english]{babel}
\usepackage{mathtools}

%\usepackage[OT2,T1]{fontenc}
%\DeclareSymbolFont{cyrletters}{OT2}{wncyr}{m}{n}
%\DeclareMathSymbol{\sha}{\mathalpha}{cyrletters}{"58}

\DeclareFontFamily{U}{wncy}{}
\DeclareFontShape{U}{wncy}{m}{n}{<->wncyr10}{}
\DeclareSymbolFont{mcy}{U}{wncy}{m}{n}
\DeclareMathSymbol{\Sh}{\mathord}{mcy}{"58} 
\DeclareMathOperator*{\argmin}{arg\,min}

\newcounter{eqn}
\renewcommand*{\theeqn}{\alph{eqn})}
\newcommand{\num}{\refstepcounter{eqn}\text{\theeqn}\;}

\makeatother
\newcommand{\vectornorm}[1]{\left|\left|#1\right|\right|}
\newcommand*\conjugate[1]{\bar{#1}}

\newtheorem{thm}{Theorem}
\newtheorem{defn}{Definition}
 %\theoremstyle{plain}
  \newtheorem{theorem}{Theorem}[section]
  \newtheorem{corollary}[theorem]{Corollary}
  \newtheorem{proposition}[theorem]{Proposition}
  \newtheorem{lemma}[theorem]{Lemma}
\newtheorem{example}[theorem]{Example}
  \newtheorem{definition}[theorem]{Definition}
  \newtheorem{conj}[theorem]{Conjecture}
 \newtheorem{condition}{Condition}
 \newtheorem{remark}[theorem]{Remark}

\newcommand{\supp}{\operatorname{supp}} 
\newcommand{\vc}[1]{{\mathbf{ #1}}}
\newcommand{\tn}{\widetilde{\nabla}_{n} }
\newcommand{\Z}{{\mathbb{Z}}}
\newcommand{\re}{{\mathbb{R}}}
\newcommand{\II}{{\mathbb{I}}}
\newcommand{\ep}{{\mathbb{E}}}
\newcommand{\pr}{{\mathbb{P}}}
\newcommand{\FF}{{\mathcal{F}}}
\newcommand{\TT}{{\mathcal{T}}}
\newcommand{\phin}{\phig{n}}
\newcommand{\phig}[1]{\phi^{(#1)}}
\newcommand{\ol}[1]{\overline{#1}}
\newcommand{\eff}{{\rm eff}}
\newcommand{\suc}{{\rm suc}}
\newcommand{\tends}{\rightarrow \infty}
\newcommand{\setS}{{\mathcal{S}}}
\newcommand{\setP}{{\mathcal{P}}}
\newcommand{\setX}{{\mathcal{X}}}
\newcommand{\nec}{{\rm nec}}
\newcommand{\bd}{{\rm bd}}
\begin{document}
\nocite{*}
\title{Bayesian Bounds, and Large Deviations}
\date{\today}
\author{Tom Kealy}
\maketitle

This is a short note, outlining some ideas between the variational Bayesian framework and Large Deviations theory. 

Following \cite{Banerjee}, we state a simple lemma (the compression lemma) which can simplify some bounds obtained in PAC Bayesian learning.

\begin{lemma}
Let \(\left(\Omega, \mathcal{F}, \pr\right)\) be a probability space, on a metric space \(\left(H, \mu\right)\). Let \(\psi\left(h\right)\) be a measurable function on \(H\), and \(P,Q\) be distributions on \(\Omega\). We have:

\begin{equation}
\ep_{Q} \left(\mf \right) - \log{\ep_{P} \left(\exp{\mf }\right)} \leq D\left(Q||P\right)
\end{equation}
futher

\begin{equation}
\sup_{\psi}{\ep_{Q} \left(\mf \right) - \log{\ep_{P} \left(\exp{\mf }\right)}} = D\left(Q||P\right)
\end{equation}
\end{lemma}
This is an elementary version of the Donsker-Varadhan formula - it's true for reasons deeper than this, but the proof is correspondingly more difficult. 
\begin{proof}
For any measurable \(\mf \), we have: 

\begin{align}
\ep{\mf} &= \ep_Q{\log{\frac{dQ}{dP} \exp{\mf} \frac{dP}{dQ} }}
\\& = D\left(Q||P\right) + \ep_Q{\log{\exp{\mf} \frac{dP}{dQ} }} 
\\& \leq D\left(Q||P\right) + \log{\ep_Q{\exp{\mf} \frac{dP}{dQ} }}
\\& = D\left(Q||P\right) + \log{\ep_P{\exp{\mf}}}
\end{align}
The supremum is achieved if we take:
\begin{equation}
\mf = \log{\frac{dQ}{dP}}
\end{equation}

\end{proof}
\begin{thebibliography}{9}

\bibitem{Banerjee}
  On Bayesian Bounds,
  Arindam Banerjee

\end{thebibliography}  
\end{document}