\documentclass[11pt]{article}
\usepackage{cite}
\usepackage[cmex10]{amsmath}
\usepackage{amsfonts,amssymb}
\usepackage[pdftex]{graphicx}
\usepackage[algoruled]{algorithm2e}
%usepackage{titlesec}
\usepackage{graphicx}

\bibliographystyle{ieeeconf}


%\usepackage{amsmath,amsfonts,amssymb,latexsym,amsthm,verbatim,a4wide}


 %\theoremstyle{plain}
  \newtheorem{theorem}{Theorem}[section]
  \newtheorem{corollary}[theorem]{Corollary}
  \newtheorem{proposition}[theorem]{Proposition}
  \newtheorem{lemma}[theorem]{Lemma}
\newtheorem{example}[theorem]{Example}
  \newtheorem{definition}[theorem]{Definition}
  \newtheorem{conj}[theorem]{Conjecture}
 \newtheorem{condition}{Condition}
 \newtheorem{remark}[theorem]{Remark}

\newcommand{\supp}{\operatorname{supp}} 
\newcommand{\vc}[1]{{\mathbf{ #1}}}
\newcommand{\tn}{\widetilde{\nabla}_{n} }
\newcommand{\Z}{{\mathbb{Z}}}
\newcommand{\re}{{\mathbb{R}}}
\newcommand{\II}{{\mathbb{I}}}
\newcommand{\ep}{{\mathbb{E}}}
\newcommand{\pr}{{\mathbb{P}}}
\newcommand{\FF}{{\mathcal{F}}}
\newcommand{\TT}{{\mathcal{T}}}
\newcommand{\phin}{\phig{n}}
\newcommand{\phig}[1]{\phi^{(#1)}}
\newcommand{\ol}[1]{\overline{#1}}
\newcommand{\eff}{{\rm eff}}
\newcommand{\suc}{{\rm suc}}
\newcommand{\tends}{\rightarrow \infty}
\newcommand{\setS}{{\mathcal{S}}}
\newcommand{\setP}{{\mathcal{P}}}
\newcommand{\setX}{{\mathcal{X}}}
\newcommand{\nec}{{\rm nec}}
\newcommand{\bd}{{\rm bd}}
\DeclareMathOperator*{\argmin}{arg\,min}
\title{Greedy Pursuit Algorithms}

\author{Thomas Kealy}

\begin{document}
\maketitle

\section*{OMP}

\begin{algorithm}[H]

 \SetKwData{Left}{left}\SetKwData{This}{this}\SetKwData{Up}{up}
 \SetKwFunction{Match}{match}\SetKwFunction{Update}{update}
 \SetKwInOut{Input}{input}\SetKwInOut{Output}{output}

 \Input{Observation vector \(\textbf{y}\), sensing matrix \(\textbf{A}\), stopping criterion \(s\)}
 initialization \\
 \(k \leftarrow 0\) \\
 \(\Lambda_0 = \emptyset\) \\
 \(\textbf{r}_k \leftarrow \textbf{y}\) \;
 \While{\(k \neq s\) or \(\textbf{r}_k \neq 0 \)}{
   \(k \leftarrow k+1\)\;
   \Match{\(h_k = A^T \textbf{r}^{k-1}\)}\;
   \(\Lambda_k = \Lambda_{k-1} \cup \argmin_j{h_k\left(j\right)} \)\;
  \Update:\;
  \(\hat{\textbf{x}} \leftarrow \textbf{A}_{\Lambda_k}^\dagger \textbf{y}\)\;
  \(\textbf{r}_k \leftarrow \textbf{y} - \textbf{A}_{\Lambda_k}^\dagger \textbf{y} \)
 }
 \Output{\(\hat{\textbf{x}} = \argmin_{x:\supp\left(x\right) \subseteq  \Lambda_k}\|y-Ax\|_2\)}
 \caption{Orthogonal Matching Pursuit}
\end{algorithm}

OMP iteratively detects the support set of the signal - proceeding by extending the estimate by a single index per iteration. The algorithm uses a matched filter to find the column which is maximally correlated with the current estimate, and solves a least-squares problem to update the residual vector.

Remarks:
\begin{itemize}
\item The residual vector at step \(k\), \(\textbf{r}_k\), can be viewed as the orthoganalisation of the observation vector against the previously chosen columns of \(\textbf{A}\).
\end{itemize}

\end{document}