\documentclass[11pt]{article}
\usepackage{amsmath, amssymb}
\usepackage{amsfonts}
\usepackage{mathrsfs}
\usepackage[arrow,matrix,curve,cmtip,ps]{xy}

\usepackage{amsthm}


\allowdisplaybreaks



\newtheorem{theorem}{Theorem}[section]
\newtheorem{lemma}[theorem]{Lemma}
\newtheorem{proposition}[theorem]{Proposition}
\newtheorem{corollary}[theorem]{Corollary}
\newtheorem{notation}[theorem]{Notation}
\newtheorem*{theorem*}{Theorem}
\theoremstyle{remark}
\newtheorem{remark}[theorem]{Remark}
\newtheorem{definition}[theorem]{Definition}
\newtheorem{example}[theorem]{Example}
\newtheorem{question}{Question}

%this has equations numbered within sections 1.1,1.2, ... 2.1,...
\numberwithin{equation}{section}


%-------------------------------------------
%       Begin Local Macros
%-------------------------------------------

\newcommand{\Z}{\mathbb{Z}}
\newcommand{\N}{\mathbb{N}}
\newcommand{\R}{\mathbb{R}}
\newcommand{\C}{\mathbb{C}}
\newcommand{\T}{\mathbb{T}}
\newcommand{\E}{\mathbb{E}}
\newcommand{\im}{\operatorname{im}}
\newcommand{\coker}{\operatorname{coker}}
\newcommand{\ind}{\operatorname{ind}}
\newcommand{\rank}{\operatorname{rank}}
\newcommand\mc[1]{\marginpar{\sloppy\protect\footnotesize #1}}
\newcommand{\twopartdef}[4]
{
	\left\{
		\begin{array}{ll}
			#1 & \mbox{if } #2 \\
			#3 & \mbox{if } #4
		\end{array}
	\right.
}        

%-------------------------------------------
%       End Local Macros
%-------------------------------------------

\begin{document}
\title{How big should a group be?}
\author{Tom Kealy}

\date{\today}


\maketitle

%%%%%%%%%%%%%%%%%%%%%%%%%%%%%%%%%%%%%%%%%%%%%%%%
\section{Introduction}
%%%%%%%%%%%%%%%%%%%%%%%%%%%%%%%%%%%%%%%%%%%%%%%%

GIven a series of probabilities \(q_i\), \(i=1\ldots n\),  \(\sum_i=1^n = K\), that each item is a set is defective, we need to choose the size of the set to be tested. That is, if we test a group of size \(S\) we get a payoff:

\begin{equation}
\theta\left(S\right) = \twopartdef {S} {\text{Group contains no defectives}} {0} {\text{Group contains a defective}}
\end{equation}

i.e. in a group of size \(S\) the payoff is: 

\begin{equation}
\theta\left(S\right) = S \prod_{i=1}^S (1-q_i)
\end{equation}

where \(\prod_{i=1}^S(1-q_i)\) is the probability that the group contains no defectives. How do we maximise \(\theta\)?

%%%%%%%%%%%%%%%%%%%%%%%%%%%%%%%%%%%%%%%%%%%%%%%%
\section{Approximation}
%%%%%%%%%%%%%%%%%%%%%%%%%%%%%%%%%%%%%%%%%%%%%%%%
Assuming that the \(q_i\) are small (say of the order \(\frac{1}{100}\) and have bounded variance, replace each \(q_i\) by the average: 

\begin{equation}
\overline{q} = \frac{1}{n} \sum_{i=1}^n q_i
\end{equation}

The payoff in this model is then: 

\begin{equation}
\theta\left(S\right) = S (1-\overline{q})^S
\end{equation}

i.e. the model is geometric (if we were flipping coins with bias \(\overline{q}\) then the number of steps until the first failure is distributed according to the geometric distribution). 

The error can be computed: 

\begin{equation}
(1-\overline{q})^S - \prod_{i=1}^S (1-q_i) \leq e^{-S\overline{q}} - e^{-\sum_{i=1}^S q_i} 
\end{equation}

Since the model is geometric, the expected number of steps until the first failure is: 

\begin{equation}
\E{Steps} = \frac{1}{\overline{q}}
\end{equation}

Because of the error, I think a reasonable choice of S is: 

\begin{equation}
S = \lfloor \frac{1}{\overline{q}} \rfloor
\end{equation}

\end{document}
